% vim: set fileencoding=utf-8 encoding=utf-8 textwidth=80
\documentclass[12pt,spanish]{article}
\renewcommand{\familydefault}{\rmdefault}
\usepackage[utf8]{inputenc}
\usepackage{fancyhdr}
\usepackage{graphicx}
\usepackage{hyperref,wasysym}
\pdfpagewidth 8.5in
\pdfpageheight 11in
\usepackage{boxedminipage}
\usepackage{array}
%\topmargin  1 cm
\setlength{\textwidth}{15.5cm}


%Header and Footer
\pagestyle{fancy}
\renewcommand{\headrulewidth}{0mm} %Para eliminar barra de header
\renewcommand{\footrulewidth}{0.4pt}
\newcommand{\tnhl}{\tabularnewline\hline}
\headheight 30pt  
\lfoot{Proyecto de Tesis}
\rfoot{\thepage}
\cfoot{}
\rhead{}
\lhead{}
\chead{\setlength{\unitlength}{1mm}
\begin{picture}(0,0)
\put(-60,0){\includegraphics[width=100mm]{logo.jpg}}
\end{picture}}




\begin{document}

\section*{PROYECTO DE TESIS}
{\huge \XBox} Doctorado en Ingeniería Informática
\\
\\
{\huge \Square} Magíster en Ciencias de la Ingeniería Informática


\begin{center}
\begin{tabular}{|%
    r @{ } %
    >{\bfseries\raggedright\hspace{0pt}} p{0.4\textwidth} |%
    >{\raggedright\hspace{0pt}}          p{0.5\textwidth} <{} |%
}\hline
   1.& Título del Proyecto de Tesis &   TituloProyecto          \tnhl
   2.& Nombre del Alumno            &   NombreApellido          \tnhl
   3.& Número de Teléfono - Celular &   +56 32 2654423          \tnhl
   4.& Correo electrónico           &   alumno@inf.utfsm.cl     \tnhl
   5.& Fecha de Ingreso al Programa &   Primer semestre de 20XX \tnhl
   6.& Pregrado \\ (Título o Grado Institución, Año) & Licenciatura en Ingeniería Informática, \\
                                      Universidad Técnica Federico Santa María, 20XX\tnhl
   7.& Profesor Guía de Tesis       & NombreApellido            \tnhl
   8.& Fecha Presentación Tema de Tesis &                       \tnhl
   9.& Fecha Aprobación Tema de Tesis   &                       \tnhl
  10.& Fecha Tentativa de Término       & Agosto de 20XX        \tnhl
  11.& Comisión de Tesis Doctoral   & \tabularnewline&&\tabularnewline&& \tnhl
\end{tabular}
\end{center}
%\normalsize

\newpage
\section*{Resumen(Abstract) (Español e Inglés)}
Debe ser suficientemente informativo, y contener una síntesis
del proyecto, sus objetivos, resultados esperados y palabras claves. 
Su extensión no debe exceder el espacio disponible.

\fbox{
\begin{minipage}[t][140mm][t]{0.9\textwidth}
Resumen:
\vspace{50 mm} %Reemplazar con texto

Palabras Claves:
\vspace{10 mm}

Abstract:
\vspace{50 mm}

Keywords:
\vspace{10 mm}
\end{minipage}
}


\newpage
\section[]{Formulación general de la problemática y propuesta de tesis}

Debe contener la exposición general del problema, identificando
claramente qué aspectos relacionados con la informática son los más
relevantes.  Además, deberá contener el marco teórico, la discusión
bibliográfica con sus referencias y, finalmente, su propuesta de
tesis.  (La extensión máxima de esta sección es de hasta 5 páginas.
En hojas adicionales incluya la lista de referencias bibliográficas
citadas)  

\subsection{Formulación de la Problemática}
\subsection{Marco Teórico y Discusión Bibliográfica}
\subsection{Formulación de la Propuesta de Tesis}
\subsection*{Referencias}

\newpage
\section[]{Hipótesis de Trabajo}

Formule las hipótesis de trabajo señalando claramente su conjetura.
( Su extensión no debe exceder el espacio disponible.

\fbox{
\begin{minipage}[t][140mm][t]{0.9\textwidth}
Reemplazar con texto
\end{minipage}
}

\newpage
\section[]{Objetivos}
\subsection{Objectivos Generales}
Su extensión no debe exceder el espacio disponible.

\fbox{
\begin{minipage}[t][70mm][t]{0.9\textwidth}
Reemplazar con texto
\end{minipage}
}


\subsection{Objectivos Específicos}
Su extensión no debe exceder el espacio disponible.

\fbox{
\begin{minipage}[t][70mm][t]{0.9\textwidth}
Reemplazar con texto
\end{minipage}
}

\newpage
\section[]{Metodología y Plan de Trabajo}
Su extensión no debe exceder el espacio disponible.


\fbox{
\begin{minipage}[t][70mm][t]{0.9\textwidth}
Reemplazar con texto
\end{minipage}
}


\newpage
\section[]{Resultados}
\subsection{Aportes y Resultados Esperados}
Su extensión no debe exceder el espacio disponible.

\fbox{
\begin{minipage}[t][70mm][t]{0.9\textwidth}
Reemplazar con texto
\end{minipage}
}

\subsection{Formas de Validación}
Su extensión no debe exceder el espacio disponible.

\fbox{
\begin{minipage}[t][70mm][t]{0.9\textwidth}
Reemplazar con texto
\end{minipage}
}


\newpage
\section[]{Recursos}
\subsection{Recursos Disponibles}
Señale medios y recursos con que cuenta el Departamento de Informática
de la UTFSM, para realizar el proyecto de tesis (libros, software,
laboratorios, etc.). Su extensión no debe exceder el espacio
disponible.

\fbox{
\begin{minipage}[t][50mm][t]{0.9\textwidth}
Reemplazar con texto
\end{minipage}
}

\subsection{Recursos Solicitados}
Señale medios y recursos no disponibles en el Departamento de
Informática de la UTFSM, necesarios para realizar el proyecto de tesis
(libros, software, laboratorios, etc.). Su extensión no debe exceder
el espacio disponible.

\fbox{
\begin{minipage}[t][50mm][t]{0.9\textwidth}
Reemplazar con texto
\end{minipage}
}


\end{document}
